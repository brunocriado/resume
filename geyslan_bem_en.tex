% LaTeX file for resume
% This file uses the resume document class (res.cls)

\documentclass[margin]{res}
\usepackage [brazil]{babel}     % Nomes e hifenação em português

\usepackage{t1enc}              % Permite digitar os acentos de forma normal
\usepackage[utf8]{inputenc}

\usepackage{hyperref} % hyperlinks

\topmargin=-0.5in  % start text higher on the page
\setlength{\textheight}{10in} % increase text height to fit resume on 1 page
\begin{document}
\name{\textit{Geyslan Gregório Bem}}

\address{Juazeiro do Norte, CE, Brazil \\
  \href{mailto:geyslan@gmail.com}{geyslan@gmail.com} \\ Phone: 55 88 99617 0441}


\begin{resume}

  \section{Summary} Graduate in Computer area having expertise in software
  development, great professional relationship and focused at work.
  With more than 10 years of experience in development and 5 using
  basically C and Assembly languages on Linux platform. Looking now
  for new challenges and opportunities that allow him to learn new
  technologies and work with new people. With great interest in
  software development (Assembly, C, Lisp, Shellscript), mainly in
  linux kernel research/development.

  As Open Source developer contributed to projects such as Linux,
  Metasploit, SLAE, uzumaki and xrasengan.

  \section{Education} Universidade Estácio de Sá, Graduate in Computer
  Programming and System Analysis.

\section{Experience}

\vspace{-0.1in}
\begin{tabbing}
  \hspace{2.3in}\= \hspace{1.7in}\= \kill % set up two tab positions
  \textbf{Open Source Developer}    \>\>\textbf{Jan 2013 - Present}\\
  \textit{Random hacking}\\
  \textbf{Kernel hacking}: Linux;
\end{tabbing}\vspace{-20pt}      % suppress blank line after tabbing
\vspace{2mm} In 2013 started studying the linux kernel, contributing
effectively with bug hunting and code improvement, resulting in more
than 70 patches (and counting)

\begin{tabbing}
  \hspace{2.3in}\= \hspace{1.7in}\= \kill % set up two tab positions
  \textbf{Tribunal Regional do Trabalho da 7ª Região}    \>\>\textbf{Oct 2012 - Present}\\
  \textit{Judiciary Technician}\\
  \textbf{Government Employee}: Manager Assistant;
\end{tabbing}\vspace{-20pt}      % suppress blank line after tabbing
\vspace{2mm} Analysis of court cases and decisions that depends on the
responsibility of how to know to enforce the federal law. And
maintenance on computer equipment, as a sport.

\begin{tabbing}
  \hspace{2.3in}\= \hspace{1.7in}\= \kill % set up two tab positions
  \textbf{Bradesco Bank}    \>\>\textbf{Jan 2007 - Jul 2012}\\
  \textit{Executive Commercial Manager}\\
  \textbf{Bank Executive};
\end{tabbing}\vspace{-20pt}      % suppress blank line after tabbing
\vspace{2mm} Coordinated 72 bank agencies. Besides the formal banking
attributions, developed excel sheets (VBA macros) to automatically
track the diary agencies results by e-mail. Before assuming that role,
was Administrative Manager in two agencies and Administrative
Supervisor.

\begin{tabbing}
  \hspace{2.3in}\= \hspace{1.7in}\= \kill % set up two tab positions
  \textbf{Diretorium Informática}    \>\>\textbf{Feb 2000 - Dec 2003}\\
  \textit{Software Engineer}\\
  \textbf{Main Technologies}: Delphi, Object Pascal, C, Paradox, Firebird,
  DBMS;
\end{tabbing}\vspace{-20pt}      % suppress blank line after tabbing
\vspace{2mm}

Member of a team responsible for designing and developing three
commercial systems:

* Developed SALE (Sistema de Acompanhamento de Logística e Estoque), a
software for control logistics and product stock. The SALE runs in
Windows platform and was developed using Delphi with Firebird.

* Assisted to improve and to correct bugs in SEV (Sistema de Estoque e
Vendas), a software for stock and sales control. Runs in Windows and
was developed using Delphi with Paradox.

* Improvement of Lockar, a movie rent system that runs in Windows.
Lockar was developed using Delphi with Paradox. It was conceived as a
commercial software but was deployed as freeware later.

Other activities involved: Operating system and network
administration, Delphi and C development, debugging, profiling, code
analysis etc.

\section{Skills Base}
\textit{Operating System}: Linux (Debian, Ubuntu, \underline{Arch
  Linux}), Windows NT/XP/Vista/7;

\textit{Network}: TCP/IP protocol suite, openssh, ftp, samba;

\textit{Progamming Languages}: C (proficient), Assembly (proficient),
Pascal (proficient), Shellscript (proficient), Emacs Lisp (prior
experience), Python (prior experience), Ruby (prior experience);

\textit{Virtualization}: VirtualBox, qemu;

\textit{Cloud}: Owncloud, Apache, MySQL, PHP;

\textit{Languages}: Portuguese (fluent), English (intermediate - OOPT
B2) and Japanese (elementary/phonetic syllabaries);

\section{Open Source Projects}
\begin{itemize}
\item \textbf{SLAE}: Security Linux Assembly Expert course
  assignments. All seven assignments accomplished with merit. This
  project gave birth to some of the smallest (and functional)
  shellcodes in the world. Smaller than those present in Metasploit.
  Eg. Tiny Shell Bind TCP and Shell Bind TCP Random Port (latter is
  unique).\newline (\url{https://github.com/geyslan/SLAE})\vspace{1mm}

\item \textbf{uzumaki}: A simple Emacs buffers cycler that optimizes
  the swapping between open buffers in this famous editor.\newline
  (\url{https://github.com/geyslan/uzumaki})\vspace{1mm}

\item \textbf{xrasengan}: An xrandr wrapper to make multi-monitor
  setup easier.\newline
  (\url{https://github.com/geyslan/xrasengan})\vspace{1mm}
\end{itemize}

\section{More Info}
\begin{itemize}
\item \textbf{Linkedin}: \url{http://www.linkedin.com/in/geyslan}
\item \textbf{Github}: \url{https://github.com/geyslan}
\item \textbf{Personal page}: \url{http://geyslan.github.io}
\item \textbf{Posts}: \url{http://hackingbits.github.io/authors/#geyslan}
\end{itemize}

\end{resume}
\end{document}
