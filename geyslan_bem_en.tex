% LaTeX file for resume
% This file uses the resume document class (res.cls)

\documentclass[margin]{res}
\usepackage [brazil]{babel}     % Nomes e hifenação em português

\usepackage{t1enc}              % Permite digitar os acentos de forma normal
\usepackage[utf8]{inputenc}

\usepackage{hyperref} % hyperlinks

\topmargin=-0.5in  % start text higher on the page
\setlength{\textheight}{10in} % increase text height to fit resume on 1 page
\begin{document}
\name{\textit{Geyslan Gregório Bem}}

\address{Juazeiro do Norte, Brazil \\ \href{mailto:geyslan@gmail.com}{geyslan@gmail.com} \\ Phone: 55 88 9617 0441 \\ Postal Code: 63034-100 }


\begin{resume}

\section{Summary}  Graduating in computer science having expertise in software development, great professional relationship
                        and focused at work. With more than 10 years of experience in development and 4 using basically C and
                        Assembly languages and Linux platform. Looking now for new challenges and opportunities that allow him to learn
                        new technologies and work with new people. With great interest in software development (Assembly, C/C++, Python,
                        Ruby, ShellScript, Lisp), mainly in linux kernel development.

                        Open Source developer contributing in projects like: Linux, OSv, Metasploit, radix, SLAE, uzumaki,
                        C in Fork Book etc.

\section{Education}	Universidade Estácio de Sá, Graduating in Computer Science.

\section{Experience}

\vspace{-0.1in}
   \begin{tabbing}
   \hspace{2.3in}\= \hspace{1.7in}\= \kill % set up two tab positions
    \textbf{Open Source Developer}    \>\>\textbf{Jan 2013 - Present}\\
    \textit{Random hacking}\\
    \textbf{Kernel hacking}: Linux and OSv;
   \end{tabbing}\vspace{-20pt}      % suppress blank line after tabbing
    \vspace{2mm}
    In 2013 started studying the linux kernel, contributing effectively with patches, bug hunting and code improvement.
    Currently studying OSv kernel and helping with patches as well.

   \begin{tabbing}
   \hspace{2.3in}\= \hspace{1.7in}\= \kill % set up two tab positions
    \textbf{Tribunal Regional do Trabalho da 7ª Região}    \>\>\textbf{Oct 2012 - Present}\\
    \textit{Judiciary Technician}\\
    \textbf{Government Employee}: Director's Assistant;
   \end{tabbing}\vspace{-20pt}      % suppress blank line after tabbing
    \vspace{2mm}
    Analysis of court cases and decisions that depends on the responsibility of how to know to enforce the federal law.
    And maintenance on computer equipment, as a sport.

   \begin{tabbing}
   \hspace{2.3in}\= \hspace{1.7in}\= \kill % set up two tab positions
    \textbf{Bradesco Bank}    \>\>\textbf{Jan 2007 - Jul 2012}\\
    \textit{Executive Commercial Manager}\\
    \textbf{Bank Executive};
   \end{tabbing}\vspace{-20pt}      % suppress blank line after tabbing
    \vspace{2mm}
    Coordinated 72 bank agencies. Besides the formal banking attributions, developed excel sheets (VBA macros) to automatically track
    the diary agencies results by e-mail. Before assuming that role, was Administrative Manager in two agencies and Administrative
    Supervisor.

   \begin{tabbing}
   \hspace{2.3in}\= \hspace{1.7in}\= \kill % set up two tab positions
    \textbf{Diretorium Informática}    \>\>\textbf{Feb 2000 - Dec 2003}\\
    \textit{Software Engineer}\\
    \textbf{Main Technologies}: Delphi, Object Pascal, C, Paradox, Firebird, DBMS;
   \end{tabbing}\vspace{-20pt}      % suppress blank line after tabbing
    \vspace{2mm}

    Member of a team responsible to design and develop three commercial systems:

      * Developed SALE (Sistema de Acompanhamento de Logística e Estoque), a software program for control logistics and
        product stock. The SALE runs in Windows platform and was developed using Delphi with Firebird.

      * Help improve and correct bugs in SEV (Sistema de Estoque e Vendas), a software program for stock and sales control.
        Runs in Windows and was developed using Delphi with Paradox.

      * Improvement of Lockar, a movie rent system that runs in Windows. Lockar was developed using Delphi with Paradox.
        It initiated as commercial software but after was deployed as freeware.

      Other activities involved: Operating system and network administration, Delphi and C development, debugging, profiling,
      code analysis etc.

\section{Skills Base} \textit{Operating System}: Linux (Debian, Ubuntu, Arch Linux), Windows NT/XP/Vista/7;

			\textit{Networkings}: TCP/IP protocol suite;

			\textit{Progamming Languages}: C/C++, Pascal, Python, ShellScript, eLisp, plus some experience in Ruby;

			\textit{Virtualization}: VirtualBox and qemu;

			\textit{Languages}: Fluent in Portuguese, Intermediate in English and Elementary in Japanese (Phonetic
                                            Syllabaries);

\section{Open Source Projects}
		\begin{itemize}
                    \item \textbf{SLAE}: Security Linux Assembly Expert course assignments. All seven assignments accomplished with
                                         merit. This project resulted in the creation of some of the smaller shellcodes of its kind
                                         in the world. Smaller than those present in Metasploit. Eg. Tiny Shell Bind TCP and Shell Bind
                                         TCP Random Port (latter is unique in its kind).\newline
                                         (\url{https://github.com/geyslan/SLAE})\vspace{1mm}

		    \item \textbf{radix}: A kernel project for academic purposes that uses grub and its multiboot support.
                                          It is maintened by radix-kernel organization.\newline
                                          (\url{https://github.com/radix-kernel/radix})\vspace{1mm}

		    \item \textbf{uzumaki}: A simple Emacs buffers cycler that optimize the swapping between open buffers in that famous
                                            editor.\newline
                                            (\url{https://github.com/geyslan/uzumaki})\vspace{1mm}

		    \item \textbf{C in Fork Book}: It is a pilot project to write a book about the C programming language addressing
                                                   beginners in computers. It will be originally written in Portuguese but with plans
                                                   to be translated into English.\newline
                                                   (\url{https://github.com/c0defellas/c.in.fork.book})
		\end{itemize}

\section{More Info}
    \begin{itemize}
        \item \textbf{Linkedin}: \url{http://www.linkedin.com/in/geyslan}
         \item \textbf{Github}: \url{https://github.com/geyslan}
         \item \textbf{Personal page}: \url{http://www.hackingbits.com}
    \end{itemize}

\end{resume}
\end{document}
